\documentclass{moderncv}
\usepackage{lmodern}
\usepackage[utf8]{inputenc}
\usepackage{verbatim}
\usepackage{setspace}

% moderncv themes
\moderncvtheme{casual}
\usepackage[left=10mm,right=15mm,top=10mm,bottom=0.8in]{geometry}
\newcommand{\hlink}[1]{\href{https://#1}{#1}}

\firstname{Saif}
\familyname{Khattak}
\email{saifkhattak.00@gmail.com}
\extrainfo{\hlink{linkedin.com/in/skhattak00/} | \hlink{github.com/saifk00} | \hlink{skhattak.dev}}
\nopagenumbers{}
\onehalfspacing % Set line spacing to 1.5

\begin{document}
\makecvtitle
\begin{comment}
\section{Skills}
\begin{minipage}{0.25\textwidth}
    \begin{itemize}
        \item Java
        \item C++
        \item Python
        \item Scala
    \end{itemize}
\end{minipage}
\begin{minipage}{0.25\textwidth}
    \begin{itemize}
        \item LLVM
        \item MLIR
        \item SYCL
        \item SQL
    \end{itemizes}
\end{minipage}
\begin{minipage}{0.25\textwidth}
    \begin{itemize}
        \item PyTorch
        \item ONNX
        \item Protobuf
        \item Verilog
    \end{itemize}
\end{minipage}
\begin{minipage}{0.25\textwidth}
    \begin{itemize}
        \item Compiler Design
        \item Parallel Execution
        \item FPGA Development
        \item ML Acceleration
    \end{itemize}
\end{minipage}
\end{comment}
\section{Overview}
\begin{center}
2+ years of experience in the acceleration of SQL queries and FPGA compilation.\\
\vspace{1mm}
Aiming to bridge the gap between the theoretically and practically computable.
\end{center}

\section{Work Experience}

\cventry{June 2024 - Ongoing}{SQL Compiler Engineer}
{Snowflake}{San Mateo CA}{}
{
    \textit{Compiler Platform Team}
    \begin{itemize}
        \item Architected migration of 20+ compiler stages to new expression framework, enabling feature teams to focus on \textbf{what} optimizations to implement rather than \textbf{how} to integrate them, improving compilation performance by 10-15\% and reducing system incident rate by 100x across 6B daily queries.
        \item Served as cross-functional technical consultant, democratizing access to compiler expertise across teams previously blocked by low-level framework complexity.
        \item Participated in quarterly support rotation, triaging and resolving 40-50 customer issues weekly in 3-person team covering query performance degradation, correctness bugs, and compilation failures across the full execution stack.
    \end{itemize}
    \textit{Data Governance Team}
    \begin{itemize}
        \item Led a project to design \textbf{new SQL syntax} enabling the application of multiple policies on a table, eliminating boilerplate code and reducing likelihood of human error when applying privacy constraints. % Implemented compiler changes in \textbf{Java} and \textbf{FoundationDB} metadata updates to support this functionality.
        \item Designed algorithms to enforce Join Policy semantics on the parse tree of a SQL query, reducing the manual effort required to sanitize data before sharing. % the whole equivalence class thing, involved talking to Principals and discussions with product team on whether people would understand it
        \item Designed optimizations on query execution plans to increase the query flexibility while maintaining privacy guarantees, reducing the amount of rewriting required for a query to satisfy constraints.  % deferment
    \end{itemize}
}

\cventry{Aug 2023 - June 2024}{FPGA Compiler Engineer}
{Intel}{Toronto ON}{}
{\begin{itemize}
    \item Enabled domain experts to describe compute kernels in high-level \textbf{SYCL} while automatically generating optimized \textbf{RTL} interfaces, democratizing access to reconfigurable hardware acceleration.
    \item Created an FPGA-specific \textbf{LLVM} optimization pass in \textbf{C++} that improved performance by 15\% across standard \textbf{OpenCL} benchmarks by using scalar evolution analysis to narrow induction variables.
    \item Debugged complex issues across the hardware-software boundary, building expertise in the full compilation stack from high-level code to hardware implementation.
\end{itemize}}

\begin{comment}
\cventry{Sep 2022 - Dec 2022}{SQL Compiler Engineer (Co-op)}{Snowflake}{San Mateo CA}{}
{\begin{itemize}
    \item Developed data privacy features at the \textbf{SQL} query engine level for Snowflake's cloud database platform.
    \item Added rules to an \textbf{ANTLR 3} grammar to enable managing data aggregation policies in \textbf{SQL}, enabling customers to share data while maintaining their users' privacy.
    \item Implemented compiler changes in \textbf{Java} to parse and generate code for applying policies to a table.
    \item Implemented changes to a custom \textbf{FoundationDB} layer to store information about policies.
\end{itemize}}
\end{comment}

\begin{comment}
Snowflake1: Aggregation Policy Parser Stuff
basically, added a new type of policy which is an SQL object
it had to be added to the parser through an ANTLR grammar
but there was an issue with the way the grammar was factored into sub-grammars
so I had to inspect generated code and eventually refactor the root rules
into subrules in separate files
    rootparser.g:
        rule -> rule1 | rule2 | rule3;
    rules1.g:
        rule1 -> [...]
        rule2 -> [...]
    rules2.g:
        rule3 -> [...]
became
    rootparser.g:
        rule -> rules1 | rules2
    rules1.g:
        rules1 -> rule1 | rule2
        rule1 -> [...]
        rule2 -> [...]
    rules2.g:
        rule3 -> [...]
and then i was able to implement the rules for that type of policy
\end{comment}

\cventry{Jan 2022 - Apr 2022}{ML Compiler Engineer (Co-op)}{Groq}{Toronto ON}{}
{\begin{itemize}
    \item Bridged the gap between ML researchers' high-level \textbf{ONNX} models and specialized hardware execution, enabling 20\% throughput improvements through automated resource utilization algorithms for tensor operations.
    \item Created optimization passes in \textbf{C++} using the \textbf{MLIR} compiler framework, applying cross-domain compilation techniques to neural network acceleration.
    \item Built end-to-end validation pipeline in \textbf{PyTorch}, enabling rapid iteration between algorithmic improvements and hardware performance measurement.
\end{itemize}}

\begin{comment}
%% Bye bye radcomm!
\cventry{Jan 2020 - Aug 2020}{Software Engineering Co-op - Embedded Systems}{RadComm Systems}{Oakville ON}{}
{\begin{itemize}
    %\item Developed software systems for controlling RadComm's radiation detection devices, which are used in recycling and medical facilities for monitoring and safety purposes
    \item Researched cutting-edge radiation analysis techniques using \textbf{GNU Octave} and \textbf{Python} for data visualization to assess development options
    % \item Implemented algorithms in \textbf{C\#} to analyze radiation patterns using the \textbf{ReactiveX} library to handle real-time data emitted by an embedded device, processing energy histograms every 100ms
    % \item Automated the device calibration process using \textbf{C\#} to allow parallel setup of many devices
\end{itemize}}
\end{comment}

%% Remove URA
\begin{comment}
\cventry{Sep 2020 - Dec 2020}{Undergraduate Research Assistant}{University of Waterloo}{Waterloo ON}{}
{\begin{itemize}
    \item Implemented novel post-quantum cryptographic algorithms in \textbf{C}
    \item Designed and implemented cache-aware optimizations resulting in 60\% speed improvement
    \item Created custom boolean matrix library for use in cryptographic algorithms
\end{itemize}}
\end{comment}

%% Sorry old pal, but we dont need ESCRYPT here anymore
\begin{comment}
    \cventry{May 2019 - Aug 2019}{Secure Software Developer}{ESCRYPT}{Waterloo ON}{}
{\begin{itemize}
    \item Implemented asynchronous process in \textbf{C++} for periodically provisioning \textbf{X.509} certificates on-vehicle, improving anonymity in the system by enabling certificate swapping
    %\item Improved signing/verifying algorithms in \textbf{C++} targeted at vehicular embedded systems
    \item Wrote ETSI-compliant tests using \textbf{GoogleTest} framework to prove functionality
\end{itemize}}
\end{comment}

% \section{Projects}
\begin{comment}
\cventry{May 2022 - Dec 2022}{Bayesian Network Inference Accelerator}{}{Scala | Chisel | Python | Verilog}{}
{\begin{itemize}
    \item Created a compiler in \textbf{Scala} to convert Bayesian network specifications into a \textbf{Verilog} module that can answer queries on the network with real-time evidence based on Markov-Chain Monte-Carlo techniques
    \item Created \textbf{Protobuf}-based specifications for model description and elaboration
    \item Created language for expressing Bayesian networks parsed using an \textbf{ANTLR 4} grammar
    \item Utilized \textbf{Chisel} to construct hardware modules dynamically and generate RTL for various backends
\end{itemize}}
\end{comment}

% \cventry{Feb 2022 - Mar 2022}{CHIP-8 Emulator}{}{C++ | SDL2 | ImGUI}{}
% {\begin{itemize}
%     \item \textbf{C++} interpreter for CHIP-8 instruction set, runs publicly available ROMs
%     \item Includes graphical and audio interface using the \textbf{SDL2} library
%     \item Implemented an assembler and disassembler to enable troubleshooting
%     \item Designed live debugger using \textbf{ImGUI} to inspect memory dumps and processor state
% \end{itemize}}

\begin{comment}
\cventry{Dec 2021 - Jan 2022}{3D Rasterized Render System}{}{C++ | CMake | OpenGL}{}
{\begin{itemize}
    \item 3D rasterized rendering system written with \textbf{OpenGL 3.3} in \textbf{C++17}
    \item Implemented mesh generation, texture loading and Phong lighting shaders
    %\item Enabled loading models from common file types based on the \textbf{Assimp} library
    %\item Uses \textbf{CMake} to allow for cross-platform development and support
\end{itemize}}
\end{comment}

% Terry Borer mentioned how everyone has these class projects
% even though we've seen them a hundred times
\begin{comment}
\cventry{Sep 2021 - Nov 2021}{Pipelined 32-Bit RISC-V Core}{}{Verilog | Verilator}{}
{\begin{itemize}
    \item Implemented RV32I spec in \textbf{Verilog} using a 5-stage pipeline design with register bypassing, simulated test programs (individual instructions and benchmark algorithms) using \textbf{Verilator} to verify design
    %\item Wrote \textbf{Python} script to run standardized RV32I instruction and benchmark tests
    %\item 5-stage pipeline with static branch prediction, register bypassing, hazard detection
\end{itemize}}
\end{comment}

% OPERATING SYSTEMS - i would rather have some Rust-based thing here
\begin{comment}
\cventry{May 2021 - Sep 2021}{Real-Time Operating System Kernel}{}{C | ARM Assembly | Keil uVision}{}
{\begin{itemize}
    \item Implements custom memory manager, EDF scheduler, IPC, and interrupt-based I/O
    \item Targets \textbf{ARM Cortex-M3} processor, implemented context switching in assembly
    \item Allows user to run custom commands through \textbf{UART} terminal
\end{itemize}}
\end{comment}

\section{Education}
\cventry{Sep 2018 - Apr 2023}{University of Waterloo}{Computer Engineering B.A.Sc}{Waterloo ON}{}
{
    Graduated with distinction.
    % Cumulative average \textbf{92\%}
    % \begin{itemize}
    %     \item Degree honours: Dean's honours list, graduated with distinction
    %     \item Scholarships and awards
    %     \begin{itemize}
    %         \item President's Research Award
    %         \item Savvas Chamberlain Scholarship
    %     \end{itemize}
    % \end{itemize}
}

\begin{comment}
    \begin{minipage}{0.3\textwidth}
\begin{itemize}
    \item Computer Architecture
\end{itemize}
\end{minipage}
\begin{minipage}{0.3\textwidth}
\begin{itemize}
    \item Operating Systems
\end{itemize}
\end{minipage}
\begin{minipage}{0.3\textwidth}
\begin{itemize}
    \item Reinforcement Learning
\end{itemize}
\end{minipage}
\end{comment}
\end{document}